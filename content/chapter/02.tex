%!TEX root = ../../main.tex

\chapter{Das Monty-Hall-Problem - Einführung}

Das Monty-Hall-Problem (oft auch als Ziegenproblem bezeichnet) ist Statistikern als Paradoxon in der Elementarwahrscheinlichkeitstheorie bekannt. Es geht dabei um die Frage, ob eine Wahl, die zunächst
zufällig unter drei a priori gleich wahrscheinlichen Möglichkeiten getroffen wurde, geändert werden sollte, wenn zusätzliche
Informationen enthüllt werden. Die Aufgabe ist ursprünglich aus der von Monty Hall modertierten Spielshow ''Lets Make a Deal'' abzuleiten. Sie fand ihren Anfang
1963 und ihr Ende 1977. In jeder Folge spielte der Host der Spielshow, Monty Hall, Spiele mit seinem Publikum. Diese Spiele variierten stark, jedoch war das Grundprinzip
häufig das Selbe. Spieler aus dem Publikum mussten sich zwischen garantierten, kleinen Gewinnen und eventuellen, durch ''Glückspiel'' gewonnene, große Preise, entscheiden.
In einer der Folgen, die häufig als die Klimax der Show beschrieben wird, wurde den Spielern aus dem Publikum drei identische Türen gezeigt. Hinter diesen drei
identischen Türen befinden sich die potenziellen Gewinne, wovon zwei Gewinne Ziegen waren. Hinter der dritten Tür war jedoch der Hauptgewinn, ein neues Auto.

Das Spiel konnte beginnen. Der Ablauf war nicht immer gleich. Ursprünglich entschied sich der Spieler für eine Tür, welche er allerdings noch nicht öffnete.
Der Host gibt dem Spieler die Möglichkeit, seine Entscheidung zu ändern, sofern dieser es wünscht. der Spieler entscheidet sich für eine Tür, welche jedoch nicht geöffnet wird. Dann öffnet der Host eine Tür für
den Spieler, von der er sicher weiß, dass sich eine Ziege dahinter befindet.

In seiner populärsten Variante, welche im Folgenden zuerst ausgeführt wird, zeigt das Monty-Hall Problem auf, wie sich die Wahrnehmung von Wahrscheinlichkeiten von der tatsächlichen Mathematik unterscheidet.
Wird einem zum ersten Mal das Problem gestellt, wählen die meisten intuitiv die ''falsche'' Option. Wird einem dann erklärt, warum man sich anders entscheiden sollte, scheint die Mathematik auf den ersten Blick paradox.
