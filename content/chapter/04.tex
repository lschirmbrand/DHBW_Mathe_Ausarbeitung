\chapter{Monty-Hall-Problem Argumente}

Grundlegend ergeben sich drei Möglichkeiten zur Auswahl für den Spieler. Er entscheidet sich, seine Auswahl zu verwerfen und die alternative Tür zu nehmen.
Weiterhin kann dem Spieler die Entscheidung unnötig vorkommen, fälschlicherweise wird oft angenommen, dass sich die Wahrscheinlichkeit nicht ändert. Zuletzt gibt es 
noch die Möglichkeit, dass der Spieler die Auswahl nicht wechselt. Er lässt die von ihm initial gewählt Tür weiterhin als seine finale Auswahl bestehen.
Im Folgenden werden alle Argumente genauer untersucht und besprochen.

    \section{Auswahl wechseln!}
    Diese Argumentation ist erfahrungsgemäß die am häufigsten getroffene und ist für die Meisten die überzeugenste Auswahl. Nehmen wir an, dass wir zu Beginn
    100 identische Türen haben. Die initiale Entscheidung bedeutet, dass die Entscheidung eine Erfolgswahrscheinlichkeit von $\frac{1}{100}$ aufweist. Das bedeutet, 
    dass sich der Preis in 99 von 100 Fällen hinter einer der anderen Türen befindet.

    \section{Wechseln macht keinen Unterschied}
    Nachdem der Host eine Tür geöffnet hat, sind zwei Türen mit der selben Wahrscheinlichkeit übrig. Allerdings entspricht dies nur der Logik, wenn der Spieler das 
    ursprüngliche Problem vergisst und die zwei verbleibenden Türen als neues Problem sieht. In diesem Fall entspricht die Wahrscheinlichkeit, trügerischerweise, in
    beiden Fällen $\frac{1}{2}$. In einer solchen Sitaution sieht der Spieler keinen Sinn in einer Entscheidung und überlässt die Entscheidung über den Ausgang 
    des Spiels, wie er annimmt, allein seinem Glück. Eine Entscheidung sei obsolet.


    \section{Auswahl nicht wechseln!}
    Argument zu nicht wechseln