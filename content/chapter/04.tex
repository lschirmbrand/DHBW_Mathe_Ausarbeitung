\chapter{Weitere Varianten}

Die in \autoref{chap:monty-hall-standard} beschriebenen Varianten unterscheiden sich nur im Verhalten des Moderators. Alle anderen Regeln sind festgelegt.
Jedoch geht aus der ursprünglichen Beschreibung von Marilyn vos Savant (\cite{Savant:1990}) zum Beispiel nicht hervor, ob der Moderator immer eine Türe öffnen und den Wechsel anbeiten muss.

Im Folgenden werden noch weitere Varianten des Monty-Hall-Problems geschildert, die von den Regeln des Monty-Hall-Standart-Problems in irgendeiner Weise abweichen.

\section{Das Monty-\textit{Fall}-Problem}

In dieser Variante wird das Öffnen der Türe durch den Moderator nicht beabsichtigt. Stattdessen - wie der Name vermuten lässt - rutscht er aus und \textit{fällt} dabei auf eine der Türen, die der Kandidat nicht ausgewählt hat und dabei aufgeht. Nur aus reinem Zufall befindet sich hinter der Türe eine Ziege. Trotzdem wird dem Kandidaten ein Wechsel angeboten.

Betrachten wir wieder den Fall, dass der Kandidat zunächst deas erste Tor auswählt. Der Moderator rutsch aus und offenbart aus versehen hinter Tor 3 eine Ziege.

\begin{equation*}
    \begin{split}
        P(M_3|G_1) &= 1 \\
        P(M_3|G_2) &= 1 \\
        P(M_3|G_3) &= 0 \\
    \end{split}
\end{equation*}

Mit dem Satz von Bayes (\autoref{equ:bayes_simpl}) erhällt man folgende Wahrscheinlichkeiten, dass das Auto hinter den jeweiligen Toren steht:

\begin{equation}
    \begin{split}
        P(G_1|M_3) & = \frac{P(M_3|G_1)}{P(M_3|G_1) + P(M_3|G_2) + P(M_3|G_3)} \\
        & = \frac{1}{1+1+0} = \frac{1}{2} \\
        P(G_2|M_3) & = \frac{1}{1+1+0} = \frac{1}{2} \\
        P(G_3|M_3) & = \frac{0}{1+1+0} = 0
    \end{split}
\end{equation}

In diesem Fall ist es also tatsächlich egal, ob man sich umentscheidet, oder bei der ersten Wahl bleibt.

(\cite{Rosenthal:2008})
