\chapter{Die Rolle des Moderators}

Aufgrund von nicht eindeutig ausformulierten Regeln befanden einige Wissenschaftler die Aufgabe für nicht eindeutig lösbar und es wurde eine Neuformuliering des
Ziegenproblems vorgeschlagen. Diese abgeänderte Version der Aufgabe wurde als das Monty-Hall-Standard-Problem bekannt. Weiterhin sollte das Problem zur gleichen
Lösung, wie die von Marilyn vos Savant vorgelegten. Zusätzlich sollen aber nun weitere Zusatzinformationen gegeben werden, welche erfahrungsbezogene Antworten ungültig
machen, und darüber hinaus soll die aktuelle Spielsituation berücksichtigt werden. (Einfügen:  Peter R. Mueser, Donald Granberg: The Monty Hall Dilemma Revisited: Understanding the Interaction of Problem Definition and Decision Making. (Memento vom 22. Juli 2012 im Internet Archive) In: University of Missouri Working Paper. 1999-06.)

Aufgeteilt in Einzelschritte ergeben sich dann die folgenden Regeln für den Spieler und den Host.
\begin{enumerate}
    \item Ein Auto und zwei Ziegen werden zufällig auf drei Tore verteilt.
    \item Zu Beginn des Spiels sind alle Tore verschlossen, sodass Auto und Ziegen nicht sichtbar sind.
    \item Der Kandidat, dem die Position des Autos völlig unbekannt ist, wählt ein Tor aus, das aber vorerst verschlossen bleibt.
    \item Fall A: Hat der Kandidat das Tor mit dem Auto gewählt, öffnet der Moderator eines der beiden anderen Tore, hinter dem sich immer eine Ziege befindet. 
    Der Moderator hat dabei die freie Wahl. Bei den nachfolgenden Lösungen werden allerdings Zusatzannahmen über die Art des Auswahlprozesses, die der Moderator
    verwendet, gemacht.
    \item Fall B: Hat der Kandidat ein tor mit einer Ziege gewählt, \textit{muss} der Moderator das Tor nehmen, hinter dem die andere Ziege steht.
    \item Der Moderator bietet dem Kandidaten an. seine Entscheidung zu überdenken und das andere ungeöffnete Tor zu wählen.
    \item Das letztendlich vom Kandidaten gewählte Tor wird geöffnet und er erhält das Objekt, dass sich hinter dem Tor befindet.
\end{enumerate}

\newpage
\textbf{Bedeutung der Zusatzannahme zum Verhalten des Moderators:}

\section{Der Moderator handelt faul}

Für die folgende Erklärung wird angenommen, dass der Spieler zu Beginn immer Tor 1 wählt. Der Moderator sei faul und beharre immer auf das gleiche Tor, sofern dieses
Tor kein Auto enthält, in diesem Fall muss der Moderator seine Faulheit überwinden und nimmt ein anderes Tor. Analog dazu funktionieren auch alle anderen Annahmen,
die das Schema einhalten.

Es handelt sich hierbei um ein Laplace-Experiment. Alle Tore haben die selbe Wahrscheinlichkeit und die zwei Ziegen können von einander unterschieden werden. 

\subsection{Tabellarische Lösung}


\begin{tabular}[h]{lccccr}
    \textbf{Fall:} & \textbf{Tor 1} & \textbf{Tor2} & \textbf{Tor3} & \textbf{Wahl des Hosts} & \textbf{Gewinn für den Spieler} \\
    1 & Auto & Ziege & Ziege & Tor 3 & Nein\\
    2 & Ziege & Auto & Ziege & Tor 3 & Ja\\
    3 & Zeige & Ziege & Auto & Tor 2, da Tor 3 Auto & Ja\\
    4 & Auto & Ziege & Ziege & Tor 3 & Nein\\
    5 & Ziege & Auto & Ziege & Tor 3 & Ja\\
    6 & Ziege & Ziege & Auto & Tor 2, da Tor 3 Auto & Ja\\
\end{tabular}

Zur Auswertung der tabellarischen Ausarbeitung muss man nun die relevanten Fälle überprüfen. Laut der zuvor festgelegten Bedingungen darf der Moderator nur Tor 3 öffnen.
Dies geschieht in den Fällen 1,2,4 und 5. Der Spieler gewinnt aber nur in zwei dieser vier Fällen. Das bedeutet die Siegeswahrscheinlichkeit beträgt $\frac{2}{4}$,
also $\frac{1}{2}$.

\newpage
\subsection{Formelle mathematische Lösung}

Durch die im Vorhinein festgelegten Bedingungen ergibt sich die folgende Situation:
Der Spieler aus dem Publikum hat das Tor 1 gewählt. Der Moderator hat daraufhin das Tor 3 geöffnet.
Somit gelten dann folgende mathematische Beziehungen unter Berücksichtigung der zuvor definierten Ereignismengen:

\begin{equation} \label{eq1}
\begin{split}
    P(G_1) = P(G_2) = P(G_3) = \frac{1}{3} \\
    P(M_3 | G_1) = 1 \\
    P(M_3 | G_2) = 1 \\ 
    P(M_3 | G_3) = 0
\end{split}
\end{equation}

Die Anwendung des zuvor erläuterten Satzes von Bayes ergibt dann für die bedingte Wahrscheinlichkeit, dass sich das Auto hinter dem Tor 2 befindet.

\begin{equation}
\begin{split}
    P(G_2 | M_3) & = \frac{P(M_3 | G_2) \cdot P(G_2)}{P(M_3 | G_1) \cdot P(G_1) +
    P(M_3 | G_2) \cdot P(G_2) + P(M_3 | G_3) \cdot P(G_3)} \\
    & = \frac{1 \cdot \frac{1}{3}}{1 \cdot \frac{1}{3} + 1 \cdot \frac{1}{3} + 1 \cdot \frac{1}{3}}
\end{split}
\end{equation}

\newpage
\section{Der Moderator handelt ausgeglichen}

Hierfür wird angenommen, dass der Spieler zu Beginn Tor 1 wählt und sich danach umentscheidet. Ist hinter dem ersten Tor das Auto enthalten, wählt der Moderator zufällig eine der beiden anderen Tore aus.
Wichtig hierbei ist, dass der Moderator diese Tore mit der gleichen Wahrscheinlichkeit auswählt und dieser somit ausgeglichen handelt. Analog dazu funktionieren auch alle anderen Annahmen,
die das Schema einhalten.

Es handelt sich hierbei um ein Laplace-Experiment. Alle Tore haben die selbe Wahrscheinlichkeit und die zwei Ziegen können von einander unterschieden werden.

\subsection{Tabellarische Lösung}

\begin{tabular}[h]{lccccr}
    \textbf{Fall:} & \textbf{Tor 1} & \textbf{Tor2} & \textbf{Tor3} & \textbf{Wahl des Hosts} & \textbf{Gewinn für den Spieler} \\
    1 & Auto & Ziege & Ziege & Tor 2 & Nein\\
    2 & Ziege & Auto & Ziege & Tor 3 & Ja\\
    3 & Zeige & Ziege & Auto & Tor 2 & Ja\\
    4 & Auto & Ziege & Ziege & Tor 3 & Nein\\
    5 & Ziege & Auto & Ziege & Tor 3 & Ja\\
    6 & Ziege & Ziege & Auto & Tor 2 & Ja\\
\end{tabular}

Zur Auswertung werden die Fälle hierfür mit einander verglichen, in dem der Moderator Tor 3 wählt. Dies ist in den Fällen 2, 4 und 5 erfüllt.
Dabei sieht man, dass der Kandidat in 2 von 3 Fällen das Auto durch den Wechsel findet. Dies ist auch für den Fall gleich, dass der Moderator Tor 2 wählt und 
der Kandidat sich für das nicht zuvor gewählte Tor entscheidet.

Somit beträgt die Wahrscheinlichkeit $\frac{2}{3}$, dass der Kandidat das Tor, hinterdem sich das Auto befindet, öffnet.

\newpage
\subsection{Formelle mathematische Lösung}
Wie zuvor beschrieben wählt der Kandidat Tor 1 aus und wechselt, nach dem öffnen eines Tores durch den Moderator, zu dem nicht zuvor nicht ausgewähltem Tor.

Folgende Ereignismengen können bei dieser Annahme definiert werden:
\begin{equation} \label{eq1}
    \begin{split}
        P(G_1) = P(G_2) = P(G_3) = \frac{1}{3} 
        \\
        \\
        P(M_3 | G_1) = \frac{1}{2} 
        \\
        \\
        P(M_3 | G_2) = 1 
        \\
        \\ 
        P(M_3 | G_3) = 0
    \end{split}
\end{equation}

Unter der Anwendung des Satzes von Bayes kann folgende Wahrscheinlichkeit bestimmt werden:

\begin{equation}
    \begin{split}
        P(G_2 | M_3) & = \frac{P(M_3 | G_2) \cdot P(G_2)}{P(M_3 | G_1) \cdot P(G_1) +
        P(M_3 | G_2) \cdot P(G_2) + P(M_3 | G_3) \cdot P(G_3)} 
        \\
        \\
        & = \frac{1 \cdot \frac{1}{3}}{\frac{1}{2} \cdot \frac{1}{3} + 1 \cdot \frac{1}{3} + 0 \cdot \frac{1}{3}} 
        \\
        \\
        & = \frac{2}{3}
    \end{split}
\end{equation}

Somit ist bewiesen, dass die anfangs Wahrscheinlichkeiten von $\frac{1}{3}$ durch das Wechseln der Tores auf ein Wahrscheinlichkeit von $\frac{2}{3}$ erhöht werden kann.  
\newpage
\section{Der Moderator handelt unausgeglichen}