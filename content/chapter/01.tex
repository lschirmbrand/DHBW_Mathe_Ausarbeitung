\chapter{A-priori-Wahrscheinlichkeit}

Die A-prior-Wahrscheinlichkeit ist in den Naturwissenschaften der Wahrscheinlichkeitswert, der aufgrund von allgmemeinem Vorwissen über die Eigenschaften des Systems
(Ein Beispiel ist der Würfel, mit seinen symmetrischen Eigenschaften) gewonnen wird. Die A-prior-Wahrscheinlichkeiten sind die Grundvorraussetzungen
bei der Berechnung der bedingten Wahrscheinlichkeit eines zusammengesetzen Ereignisses und beim bayesschen Wahrscheinlichkeitsbegriff \footnote[1]{}.
Die älteste Methode für die Bestimmung von A-prior-Wahrscheinlichkeiten stammt von Laplace. Sofern es keinen keinen expliziten Grund gibt, die ursprünglich
offensichtliche Annahme zu ändern, wird allen Elementarereignissen die gleiche Wahrscheinlichkeit zugeordnet. \cite[S. 80f]{Pap:1995}
Nimmt man das oben genante Beispiel eines Würfels ergibt sich die folgende Wahrscheinlichkeitsverteilung:

\begin{itemize} 
    \item Zahl 1: Wahrscheinlichkeit = $\frac{1}{6}$
    \item Zahl 2: Wahrscheinlichkeit = $\frac{1}{6}$
    \item Zahl 3: Wahrscheinlichkeit = $\frac{1}{6}$
    \item Zahl 4: Wahrscheinlichkeit = $\frac{1}{6}$
    \item Zahl 5: Wahrscheinlichkeit = $\frac{1}{6}$
    \item Zahl 6: Wahrscheinlichkeit = $\frac{1}{6}$
\end{itemize}

Dies ist allerdings nur der Fall, solange man keinen Grund hat anzunehmen, dass der Würfel manipuliert sei. Es handelt sich also um Elementarereignisse, der
alle dieselbe Wahrscheinlichkeit zugeordnet sind.