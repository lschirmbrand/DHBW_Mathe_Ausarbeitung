\chapter{Einführungen}

\section{A-priori-Wahrscheinlichkeit}

Die A-priori-Wahrscheinlichkeit ist in den Naturwissenschaften der Wahrscheinlichkeitswert, der aufgrund von allgmemeinem Vorwissen über die Eigenschaften des Systems
(Ein Beispiel ist der Würfel, mit seinen symmetrischen Eigenschaften) gewonnen wird. Die A-priori-Wahrscheinlichkeiten sind die Grundvorraussetzungen
bei der Berechnung der bedingten Wahrscheinlichkeit eines zusammengesetzen Ereignisses und beim bayesschen Wahrscheinlichkeitsbegriff \footnote{to be done}.
Die älteste Methode für die Bestimmung von A-priori-Wahrscheinlichkeiten stammt von Laplace. Sofern es keinen keinen expliziten Grund gibt, die ursprünglich
offensichtliche Annahme zu ändern, wird allen Elementarereignissen die gleiche Wahrscheinlichkeit zugeordnet. \cite[S. 80f]{Pap:1995}

Nimmt man das oben genante Beispiel eines Würfels ergibt sich die folgende Wahrscheinlichkeitsverteilung:

\begin{itemize}
    \item Zahl 1: Wahrscheinlichkeit = $\frac{1}{6}$
    \item Zahl 2: Wahrscheinlichkeit = $\frac{1}{6}$
    \item Zahl 3: Wahrscheinlichkeit = $\frac{1}{6}$
    \item Zahl 4: Wahrscheinlichkeit = $\frac{1}{6}$
    \item Zahl 5: Wahrscheinlichkeit = $\frac{1}{6}$
    \item Zahl 6: Wahrscheinlichkeit = $\frac{1}{6}$
\end{itemize}

Dies ist allerdings nur der Fall, solange man keinen Grund hat anzunehmen, dass der Würfel manipuliert sei. Es handelt sich also um Elementarereignisse, der
alle dieselbe Wahrscheinlichkeit zugeordnet sind.

\section{Satz von Bayes}

Der Satz von Bayes ist ein mathematischer Satz, der aus der Wahrscheinlichkeitstheorie stammt. Er beschreibt die Berechnung bedingter
Wahrscheinlichkeiten\footnote{Die Wahrscheinlichkeit des Eintretens eines Ereignisses A unter der Bedingung, dass das Einteten eines anderen Ereignisses B bereits bekannt ist.}.
Der Satz ist nach dem englischen Mathematiker Thomas Bayes benannt und wird auch als Formel von Bayes oder als Bayes-Theorem bezeichnet. \cite[S.411f]{Papula:2014}

\subsection{Formel}

Für zwei Ereignisse\footnote{Ein Ereignis ist in der Statistik ein Teil einer Menge von Ergebnissen eines Zufallexperiments, dem eine Wahrscheinlichkeit zugeordnet
    werden kann.} \textit{A} und \textit{B} mit \textit{P(B) > 0} lässt sich die Wahrscheinlichkeit von \textit{A} unter der Bedingung, dass \textit{B} eingetreten
ist, durch die Wahrscheinlichkeit von \textit{B} unter der Bedingung, dass \textit{A} eingetreten ist, errechnen:

\begin{equation}
    P(A | B) = \frac{P(B | A) \cdot P(A)}{P(B)}
\end{equation}

Hierbei ist

\begin{itemize}
    \item $P(A | B)$ die bedingte Wahrscheinlichkeit des Ereignisses $A$ unter der Bedingung, dass $B$ eingetreten ist,
    \item P(B | A) die bedingte Wahrscheinlichkeit des Ereignisses $B$ unter der Bedingung, dass $A$ eingetreten ist,
    \item $P(A)$ die A-priori-Wahrscheinlichkeit des Ereignisses $A$ und
    \item $P(B)$ die A-priori-Wahrscheinlichkeit des Ereignisses $B$.
\end{itemize}

Bei endlich vielen Ereignissen lautet der Satz von Bayes:

Wenn $A_i, i = 1,..., N$ eine Zerlegung der Ergebnismenge in disjunkte Ereignisse ist, gilt für die A-posteriori-Wahrscheinlichkeit\footnote{TOBEDONE}
$P(A_i | B)$

\begin{equation} \label{equ:bayes}
    P(A_i | B) = \frac{P(B | A_i) \cdot P(A_i)}{P(B)} \underbrace{= \frac{P(B | A_i) \cdot P(A_i)}{\sum_{j = 1}^{N} P(B | A_j) \cdot P(A_j)}}_{Marginalisierung}
\end{equation}

\cite[S.411f]{Papula:2014}

Bei den Betrachtungen des Monty-Hall-Problems sind die A-priori-Wahrscheinlichkeiten $P(A_1) = P(A_2) = ... = P(A_N) = P(A)$ alle gleich. Dadurch lässt sich \autoref{equ:bayes} vereinfachen:

\begin{equation} \label{equ:bayes_simpl}
    P(A_i | B)  = \frac{P(B | A_i) \cdot P(A)}{\sum_{j=1}^{N} P(B | A_j) \cdot P(A)}  = \frac{P(B|A_i)}{\sum_{j=1}^{N} P(B | A_j)}
\end{equation}
